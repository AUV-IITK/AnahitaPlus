\href{https://dev.mcgillrobotics.com/job/ros-teledyne-navigator/job/master}{\tt !\mbox{[}status\mbox{]}}

This R\+OS package configures and communicates with the Teledyne Navigator Doppler Velocity Log (D\+VL). {\bfseries This has only been tested on R\+OS Kinetic and Melodic over R\+S232.}

You must clone this repository as {\ttfamily teledyne\+\_\+navigator} into your {\ttfamily catkin} workspace\textquotesingle{}s {\ttfamily src} directory\+:


\begin{DoxyCode}
roscd
cd src
git clone https://github.com/mcgill-robotics/ros-teledyne-navigator.git teledyne\_navigator
\end{DoxyCode}


Before proceeding, make sure to install all the dependencies by running\+:


\begin{DoxyCode}
rosdep update
rosdep install teledyne\_navigator
\end{DoxyCode}


You {\bfseries must} compile this package before being able to run it. You can do so by running\+:


\begin{DoxyCode}
catkin\_make
\end{DoxyCode}


from the root of your workspace.

To run, simply connect the D\+VL over R\+S232 and launch the package with\+:


\begin{DoxyCode}
roslaunch teledyne\_navigator teledyne\_navigator.launch port:=</path/to/dvl>
\end{DoxyCode}


The following run-\/time R\+OS launch arguments are available\+:


\begin{DoxyItemize}
\item {\ttfamily port}\+: Serial port to read from, default\+: {\ttfamily /dev/dvl}.
\item {\ttfamily baudrate}\+: Serial baud rate, default\+: {\ttfamily 9600}.
\item {\ttfamily timeout}\+: Serial read timeout in seconds, default\+: {\ttfamily 1.\+0}.
\item {\ttfamily frame}\+: {\ttfamily tf} frame to stamp the messages with, default\+: {\ttfamily dvl}.
\end{DoxyItemize}

The package will keep trying to connect to the D\+VL until it is successful.

The {\ttfamily teledyne\+\_\+navigator} node will output to the following R\+OS topic\+:


\begin{DoxyItemize}
\item {\ttfamily $\sim$ensemble}\+: D\+VL scan data as an {\ttfamily Ensemble} message.
\end{DoxyItemize}

This driver is limited to the P\+D5 output format and the earth-\/coordinate frame (E\+NU). You may change other parameters as you please using T\+R\+DI Toolz and save them with {\ttfamily CK}, but the {\ttfamily PD} and {\ttfamily EX} commands will be overridden.

Contributions are welcome. Simply open an issue or pull request on the matter, and it will be accepted as long as it does not complicate the code base too much.

As for style guides, we follow the R\+OS Python Style Guide for R\+O\+S-\/specifics and the Google Python Style Guide for everything else.

We use \href{https://github.com/google/yapf}{\tt Y\+A\+PF} for all Python formatting needs. You can auto-\/format your changes with the following command\+:


\begin{DoxyCode}
yapf --recursive --in-place --parallel .
\end{DoxyCode}


We also use \href{https://github.com/fkie/catkin_lint}{\tt catkin\+\_\+lint} for all {\ttfamily catkin} specifics. You can lint your changes as follows\+:


\begin{DoxyCode}
catkin lint --explain -W2 .
\end{DoxyCode}


See \mbox{[}L\+I\+C\+E\+N\+SE\mbox{]}(L\+I\+C\+E\+N\+SE). 